\begin{resumo}

Atualmente, algumas organizações da Administração Pública Federal iniciam investimentos para adotar contratações de serviços de desenvolvimento de \textit{software} utilizando métodos ágeis. Nestes processos de contratação percebeu-se a lacuna sobre como aferir a qualidade interna do produto recebido. O objetivo deste trabalho foi propor uma solução automatizada, apoiada em um ambiente de \textit{data warehousing}, com intuito de aferir cenários de limpeza de código-fonte, que são indicadores de pedaços de código-fonte não coesos e com problemas de design. Visando a validação empírica da solução automatizada, realizou-se um estudo de caso no qual foi avaliado as 24 releases mensais do Sistema Integrado de Gestão e Conhecimento (SIGC) do IPHAN no qual foram identificados 397 cenários de limpeza de código-fonte em 914 classes, em um total de mais 300.000 linhas de código-fonte na 24ª release do SIGC. 

\end{resumo}
