\begin{resumo}

Atualmente, algumas organizações da Administração Pública Federal iniciam investimentos para adotar contratações de serviços de desenvolvimento de \textit{software} utilizando métodos ágeis. Nesse contexto, percebeu-se a oportunidade de melhorar a capacidade de organizações públicas em aferir a qualidade do produto de software contrato. O objetivo deste trabalho foi propor uma solução automatizada, apoiada em um ambiente de \textit{data warehousing}, que pudesse aferir a qualidade do código-fonte de forma a apoiar a aceitação ou rejeição do produto por parte da contratante. Visando a validação empírica da solução automatizada, realizou-se um estudo de caso no qual foram avaliadas as 24 releases mensais do Sistema Integrado de Gestão e Conhecimento (SIGC) do IPHAN no qual foram avaliadas mais de 39.000 linhas de código-fonte.
{
\\
\\
\textbf{Palavras-chave:} Código-fonte, contratações, métodos ágeis, IN04, qualidade do código-fonte, qualidade interna do produto, código-limpo, estudo de caso, administração pública federal.}
\end{resumo}

