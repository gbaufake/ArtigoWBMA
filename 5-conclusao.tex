\section{Conclusões e Trabalhos Futuros}


Com relação a execução do estudo de caso, foi possível mostrar que a automação do processo de medição da qualidade interna do produto de software (objetivo específico OE1)  é possível de ser realizada de maneira contínua e automatizada por meio do ambiente de \textit{data warehousing}, pois além da automação da coleta de um volume de dados de díficil verificação manual (39.000 linhas de código-fonte e 914 classes na 24ª release), foram automatizados desde a interpretação dos cenários de limpeza de código-fonte, o cálculo da taxa de aproveitamento das oportunidades de melhoria do código-fonte, até a exibição dos dados em tabelas e gráficos.

Com os resultados apresentados na Seção \ref{sec:resultados}, verifica-se que uma outra contribuição ambiente de \textit{data warehousing} foi alcançada por meio das consultas OLAP que permitiram flexibilidade necessária para se avaliar tanto em nível de projeto (objetivo específico OE2), quanto em nível local (objetivo específico OE3) a saúde do projeto com relação a qualidade do código-fonte. Por meio de um ambiente de \textit{data warehousing} semelhante ao apresentado, os fiscais técnicos dos contratos podem contar com mecanismos que venham automatizar a atividade de aferição da qualidade do código-fonte, de forma seja possível definir níveis de aceitação de produto.
Como esse estudo ocorreu na fase final do contrato, não foi possível analisar a eficiência e eficácia da medição do aproveitamento de oportunidades de melhoria de código-fonte. Entretanto, com a solução apresentada, foi possível medir o comportamento da taxa de aproveitamento de oportunidades de melhoria do código-fonte. Porém, a partir dela, não foi possível fazer conslusões mais objetivas de forma a auxiliar o fiscal técnico quanto a aceitação/rejeição do produto contratado, atendendo assim, parcialmente ao objetivo específico OE4.

<<<<<<< HEAD
Os resultados deste trabalho sugerem que o uso do ambiente proposto pode auxiliar a atividade de aferição da qualidade do produto por parte do fiscal técnico de contratos, além de auxiliá-lo a ter uma atitude propositiva ao apresentar os argumentos questionando a qualidade interna do software desenvolvido. Entretanto, trata-se de um estudo preliminar, que carece de maior aprofundamento e análises.

A partir da análise de uma amostragem de projetos, estatisticamente significativa, estudos futuros poderiam, por exemplo: identificar valores aceitáveis de nível de qualidade interna e identificar valores de referência para a taxa de oportunidade de melhoria de código-fonte.
=======
Os resultados deste trabalho sugerem que o uso do ambiente proposto pode auxiliar a atividade de aferição da qualidade do produto por parte do fiscal técnico de contratos, pois é possível com a utilização do ambiente, digirir as atividades de refatoração para as classes que mais possuem cenários de limpeza, como se mostra na Figura \ref{fig:worst-10-cenarios}, bem como ter uma visão global da evolução da qualidade do código-fonte ao longo do desenvolvimento do contrato. Como trabalho futuro, recomenda-se investigação quanto a eficácia e eficência da abordagem proposta, além da replicação deste estudo em outros casos, de forma a aumentar o poder de generalização das conclusões.
>>>>>>> FETCH_HEAD
