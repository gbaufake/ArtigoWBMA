\section{Conclusões e Trabalhos Futuros}

Considerando o objetivo de automatizar o processo de medição da qualidade interna do produto de software (objetivo específico OE1) como divisor entre incorporação da qualidade do código-fonte como critério de aceitação do produto, acredita-se que o ambiente de \textit{data warehousing} pode configurar uma boa solução para automatizar a medição de qualidade do código-fonte, pois foram automatizados desde a extração das métricas de código-fonte, a interpretação dos cenários de limpeza de código-fonte, o cálculo da taxa de aproveitamento das oportunidades de melhoria do código-fonte até a exibição dos dados em tabelas e gráficos.

Com os resultados apresentados na Seção \ref{sec:resultados}, verifica-se que uma outra contribuição ambiente de \textit{data warehousing} foi alcançada por meio das consultas OLAP que permitiram flexibilidade necessária para se avaliar tanto em nível de projeto (objetivo específico OE2), quanto em nível local (objetivo específico OE3) a saúde do projeto com relação a qualidade do código-fonte. Por meio de um ambiente de \textit{data warehousing} semelhante ao apresentado, os fiscais técnicos dos contratos podem contar com mecanismos rápidos de auditoria sobre a qualidade do código-fonte, de forma seja possível atribuir níveis de aceitação com relação a qualidade do código-fonte. 

Cabe ressaltar que o IPHAN não realizou análise das métricas de código-fonte e nem pediu a contratada que documentasse as atividades de refatoração de código-fonte ao longo da execução do contrato, logo não se pôde aprofundar nas investigações das duas hipóteses levantadas (na Seção \ref{sec:resultados}) com relação ao comportamento da taxa de aproveitamento de oportunidades de melhoria do código-fonte (objetivo OE4). Como trabalho futuro, recomenda-se que essa investigação das hipóteses seja feita nos órgãos citados pelo Acórdão no 2314/2013 \cite{TCU:2013}, em que seja possível acompanhar as atividades de refatoração do código-fonte ao longo do desenvolvimento de software.
