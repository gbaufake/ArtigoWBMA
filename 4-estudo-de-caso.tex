\section {Estudo de Caso}

Visando a validação da solução apresentada na Seção \ref{sec:solucao}, foram consultadas as instituições citadas pelo Acórdão no 2314/2013 \cite{TCU:2013}. Dentre elas, o Instituto do Patrimônio Histórico e Artístico Nacional (IPHAN) que é autarquia da administração pública federal responsável pela gestão de diversos processos de preservação do patrimônio cultural, permitiu o acesso ao processo, documentos e o código-fonte Sistema Integrado de Gestão do Conhecimento (SICG), um dos primeiros softwares desenvolvidos sob um contrato de terceirização de serviços com a utilização de metodologias ágeis.

O Sistema Integrado de Gestão do Conhecimento (SICG) teve como objetivo automatizar o processo de trabalho decorrente da metodologia de inventário, cadastro, normatização, fiscalização, planejamento e análise e gestão do patrimônio material. Esta solução de software foi construído na linguagem Java com a utilização de\textit{frameworks} conhecidos no mercado, como por exemplom VRaptor e Hibernate, durante 24 \textit{releases} mensais.

\subsection{Protocolo do Estudo de Caso}
Seguindo a metodologia proposta por Wohlin \cite{wohlin2012experimentation}, foi construído um protocolo de estudo de caso que foi baseado em Brereton \cite{brereton2008using} quanto aos aspectos gerais e Yin \cite{yin2011applications} com relação as ameaças à validade do estudo. 

O objeto do estudo de caso é a avaliação em ambiente de \textit{data warehousing} a deteceção dos cenários de Limpeza de código-fonte e o comportamento taxa de aproveitamento de oportunidades de melhoria de código-fonte de forma que o processo de medição da qualidade do código-fonte possa servir como um critério de aceitação do produto por parte do fiscal do técnico do contrato de terceirização do produto de desnvolvimento de software. A fonte de coleta dos dados, para este estudo de caso, é o código-Fonte do Sistema Integrado de Gestão do Conhecimento.

Com relação as ameaças citadas por Yin, a validade de construção pôde ser obtida com utilização da abordagem GQM para selecionar indicadores que represetem a realide conhecida. Com relação a validade interna do estudo está garantida quando se consegue observar as relações causais e todos os elementos que as compõem. No presente estudo de caso, a taxa de aproveitamento de oportunidades de refatoração é diretamente dependente da quantidade de cenários de limpeza identificados desde que o número de classes permaneça o mesmo. Quanto a validade externa, destaca-se que a utilização de um estudo de caso não é suficiente para generalizar os resultados dele obtidos, sendo necessário a utilização de estudo em múltiplos casos, a fim de comprovar resultados genéricos \cite{yin2011applications}. Com relação a confiabilidade, a partir da documentação da implementação do ambiente de \textit{Data Warehousing} conjuntamente com o protocolo de estudo de caso e as bases de dados das métricas de código-fonte, garante-se a repetição do estudo de caso e por conseguinte a confiabilidade.

\subsection{Execução e Resultados do Estudo de Caso}
\label{sec:resultados}
Para analisar os cenários de limpeza de código-fonte, foram extraídas as métricas de código-fonte de cada classe e analisadas conforme a Tabela \ref{tab:cenarios}. Para cada \textit{release}, foram identificados cenários de limpeza de código-fonte conforme a Figura \ref{fig:cenarios-release}.


\begin{figure}[ht!]
\centering
\includegraphics[keepaspectratio=true,scale=0.43]{figuras/total-cenario-tipo.eps}
\caption{Total de Cenários de Limpeza de Código-Fonte identificados por cenário e \textit{Release}}
\label{fig:cenarios-release}
\end{figure}
\FloatBarrier


Realizando uma consulta OLAP de consolidação, obteve-se o número total de cenários de limpeza por cada uma das releases de software analisadas tal como se observa na Figura \ref{fig:cenarios-total}.

\begin{figure}[ht!]
\centering
\includegraphics[keepaspectratio=true,scale=0.45]{figuras/total-cenarios-release.eps}
\caption{Total de Cenários de Limpeza de Código-Fonte por Release}
\label{fig:cenarios-total}
\end{figure}
\FloatBarrier

Conforme é possível observar nas Figuras \ref{fig:cenarios-release} e \ref{fig:cenarios-total}, foram detectados mais cenários de limpeza de código-fonte dos tipos \textbf{Complexidade Estrutural}, que corresponde entre 55\% a 68\% da quantidade total de cenários identificados; \textbf{Classe Pouco Coesa} e \textbf{Interface dos Métodos} respectivamente. Os três Cenários de Limpeza com menor número de incidências foram \textbf{Classe com Muita Exposição}, \textbf{Classe com Muitos Filhos} e \textbf{Classe com Métodos Muito Grande e/ou com muitos condicionais}.


Considerando que é importante conhecer os principais focos de incidência de limpeza de código-fonte, pois essas são as classes que mais apresentam problemas com relação a código-limp, identificou-se, como se mostra na Figura \ref{fig:worst-10-cenarios}, as 10 classes que apresentaram a maior quantidade de cenários de limpeza.

\begin{figure}[ht!]
\centering
\includegraphics[keepaspectratio=true,scale=0.55]{figuras/10-best.eps}
\caption{As 10 classes com maior número identificado de Cénarios de Limpeza}
\label{fig:worst-10-cenarios}
\end{figure}
\FloatBarrier

Com a quantidade de classes e o total de cenários de limpeza de código-fonte, foi possível calcular a Taxa de Aproveitamento de Oportunidade de Melhoria de Código-Fonte por cada release do software conforme se mostra na Figura \ref{fig:taxa-cenarios}.

\begin{figure}[H]
\centering
\includegraphics[keepaspectratio=true,scale=0.38]{figuras/taxa-parcial.eps}
\caption{Taxa de Aproveitamento de Oportunidades de Melhoria de Código-Fonte}
\label{fig:taxa-cenarios}
\end{figure}
\FloatBarrier

Observa-se, por meio da Figura \ref{fig:taxa-cenarios}, que o valor da Taxa de Aproveitamento de Oportunidade de Melhoria de Código-Fonte possui uma tendência entre 0,4 a 0,5. Este fato pode indicar duas hipóteses: a primeira de que o projeto cresceu em uma taxa muito maior que a quantidade de cenários de limpeza, indicando assim uma estabilidade na complexidade do projeto; ou que não foram promovidas atividades de melhoria de código-fonte ao longo das 24 releases.
